% Sample article for Combinatorial Theory.
%
% EJC papers *must* begin with the following two lines. 
\documentclass[12pt]{article}
\usepackage{ct}

% GEOMETRY
% Please remove all commands that change parameters such as
%    margins or page sizes.  The style file sets them

% PACKAGES
% Packages amssymb, amsthm and hyperref are already loaded. 
% We recommend also these packages for mathematics and images:
\usepackage{amsmath,graphicx}

% THEOREM ENVIRONMENTS
% Theorem-like environments that are declared in the style file are:
% theorem, lemma, corollary, proposition, fact, observation, claim,
% definition, example, conjecture, open, problem, question, remark, note

% CHARACTER CODES
% Please do not use non-ascii characters in this file, but instead use
% the LaTex macros for characters with diacritical marks, such as
% G\"{o}del, R\'{e}nyi, Erd\H{o}s.  Don't use the package "babel".
% Note that this is the opposite of the rule for the article metadata
% that you enter on the web page; sorry for the confusion!

% DATES
% Give the submission and acceptance dates in the format shown.
% The editors will insert the publication date in the third argument.
\dateline{Jan 1, 2020}{Jan 2, 2020}{TBD}

% SUBJECT CODES
% Give one or more subject codes separated by commas.
% Codes are available from http://www.ams.org/mathscinet/freeTools.html
\MSC{05C88, 05C89}

% COPYRIGHT NOTICE
% Uncomment exactly one of the following copyright statements.  Alternatively,
% you can write your own copyright statement subject to the approval of the journal.
% See https://creativecommons.org/licenses/ for a full explanation of the 
% Creative Commons licenses.
%
% We strongly recommend CC BY-ND or CC BY. Both these licenses allow others
% to freely distribute your work while giving credit to you. The difference between
% them is that CC BY-ND only allows distribution unchanged and in whole, while
% CC BY also allows remixing, tweaking and building upon your work.  
%
%    One author:  ==========================
%\Copyright{The author.}
%\Copyright{The author. Released under the CC BY license (International 4.0).}
%\Copyright{The author. Released under the CC BY-ND license (International 4.0).}
%    More than one author: ===================
%\Copyright{The authors.}
%\Copyright{The authors. Released under the CC BY license (International 4.0).}
\Copyright{The authors. Released under the CC BY-NC license (International 4.0).}

% TITLE
% If needed, include a line break (\\) at an appropriate place in the title.
\title{An elementary proof\\ of the reconstruction conjecture}

% AUTHORS
% Input author, affiliation, address and support information as follows;
% The address should include the country, but does not have to include
%    the street address. Give at least one email address.

    \newcommand\emailfont{\sffamily}
    \newcommand\emailampersat{\small@}

    \catcode`\_=11\relax
    \def\_email#1@#2\q_nil{%
      \href{mailto:#1@#2}{{\emailfont #1\emailampersat #2}}
    }    
   \newcommand\email[1]{\_email #1\q_nil}
    \catcode`\_=8\relax
  
  

  
    
\author[1]{First Author\thanks{Supported by NASA grant ABC123.}}
\author[2,3]{Some Second Author}
\author[3]{Some Third Author}

\affil[1]{Department of Inconsequential Studies, Solatido College, North Kentucky, U.S.A.\newline
\email{fa@solatido.edu}%
}
\affil[2,3]{School of Hard Knocks, University of Western Nowhere, Somewhere, Australia\newline
\email{\{ssa,sta\}@uwn.edu.au}}



\begin{document}

\maketitle

% ABSTRACT
% E-JC papers must include an abstract. The abstract should consist of a
% succinct statement of background followed by a listing of the
% principal new results that are to be found in the paper. The abstract
% should be informative, clear, and as complete as possible. Phrases
% like "we investigate..." or "we study..." should be kept to a minimum
% in favor of "we prove that..."  or "we show that...".  Do not
% include equation numbers, unexpanded citations (such as "[23]"), or
% any other references to things in the paper that are not defined in
% the abstract. The abstract may be distributed without the rest of the
% paper so it must be entirely self-contained.  Try to include all words
% and phrases that someone might search for when looking for your paper.

\begin{abstract}
  The reconstruction conjecture states that the multiset of 
  vertex-deleted subgraphs of a graph determines the graph, provided
  it has at least 3 vertices.  This problem was independently introduced
  by Stanis\l aw Ulam (1960) and Paul Kelly (1957). In this paper,
  we prove the conjecture by elementary methods. 
  It is only necessary
  to integrate the Lenkle potential of the Broglington manifold over
  the quantum supervacillatory measure in order to reduce the set of
  possible counterexamples to a small number (less than a trillion).
  A simple computer program that implements Pipletti's classification
  theorem for torsion-free Aramaic groups with simplectic socles can
  then finish the remaining cases.
% KEYWORDS (optional)
\keywords{Broglington manifolds}
\end{abstract}

\section{Introduction}

The reconstruction conjecture states that the multiset of unlabeled
vertex-deleted subgraphs of a graph determines the graph, provided it
has at least three vertices.  This problem was independently introduced
by Ulam~\cite{Ulam} and Kelly~\cite{Kelly}.  The reconstruction
conjecture is widely studied
\cite{Bollobas,FGH,HHRT,KSU,Stockmeyer,WS} and is very interesting
because it is. See \cite{BH} for more about the
reconstruction conjecture.

\begin{definition} 
  A graph is \emph{fabulous} if \emph{rest of definition here}.
\end{definition}

\begin{theorem}\label{Thm:FabGraphs}
  All planar graphs are fabulous.
\end{theorem}

\begin{proof}
  Suppose on the contrary that some planar graph is not fabulous.
  Then we have a contradiction.
\end{proof}

%%%%%%%%%%%%%%%%%%%%%%%%%%%%%%%%%%%%%%%%%%%%%%%%%%%
\section{Broglington Manifolds}

This section describes background information about Broglington
Manifolds.

\begin{lemma}\label{lem:Technical}
  Broglington manifolds are abundant.
\end{lemma}

\begin{proof}
  A proof is given here.
\end{proof}

%%%%%%%%%%%%%%%%%%%%%%%%%%%%%%%%%%%%%%%%%%%%%%%%%
\section{Proof of Theorem~\ref{Thm:FabGraphs}}

In this section we complete the proof of Theorem~\ref{Thm:FabGraphs}.

\begin{proof}[Proof of Theorem~\ref{Thm:FabGraphs}]
Let $G$ be a graph. We have
 % The align environment for multi-line equations is defined in the amsmath package.
  \begin{align}
    |X| &= a+b+c \nonumber\\
         &= \alpha\beta\gamma.
  \end{align}
  This completes the proof of Theorem~\ref{Thm:FabGraphs}.
\end{proof}

\begin{figure}[ht]
  \centering
    % Use \includegraphics to import figures; for example 
    %      \includegraphics[scale=0.6]{filename}
  \caption{Here is an informative figure.\label{fig:InformativeFigure}}
\end{figure}

%%%%%%%%%%%%%%%%%%%%%%%%%%%%%%%%%%%%%%%%%%%%%%%%%
\subsection*{Acknowledgements}

Thanks to Professor Qwerty for suggesting the proof of
Lemma~\ref{lem:Technical}.

%BIBLIOGRAPHY
% You do not have to use the same format for your references, but 
%    include everything in this file.  Don't use natbib please.
% If you use BibTeX to create a bibliography, copy the .bbl file into here.
% \newblock is optional (it adds a little space)

\begin{thebibliography}{99}

\bibitem{Bollobas} B. Bollob{\'a}s. \newblock Almost every
  graph has reconstruction number three. \newblock \emph{J. Graph Theory},
  14(1):1--4, 1990.

\bibitem{BH} J.~A. Bondy and R. Hemminger,
\newblock Graph reconstruction---a survey.
\emph{J. Graph Theory}, 1:227--268, 1977. \doi{10.1002/jgt.3190010306}.

\bibitem{FGH} J.~Fisher, R.~L. Graham, and F.~Harary. \newblock A
  simpler counterexample to the reconstruction conjecture for
  denumerable graphs. \newblock \emph{J. Combinatorial Theory, Ser. B},
  12:203--204, 1972.

\bibitem{HHRT} E. Hemaspaandra, L.~A. Hemaspaandra,
  S.~P. Radziszowski, and R. Tripathi. \newblock
  Complexity results in graph reconstruction. \newblock \emph{Discrete
    Appl. Math.}, 155(2):103--118, 2007.

\bibitem{Kelly} P.~J. Kelly. \newblock A congruence theorem for
  trees. \newblock \emph{Pacific J. Math.}, 7:961--968, 1957.

\bibitem{KSU} M. Kiyomi, T. Saitoh, and R. Uehara.
  \newblock Reconstruction of interval graphs. \newblock In 
    \emph{Computing and combinatorics}, volume 5609 of
    \emph{Lecture Notes in Comput. Sci.}, pages 106--115. Springer, 2009.

\bibitem{Stockmeyer} P.~K. Stockmeyer. \newblock The falsity of the
  reconstruction conjecture for tournaments. \newblock \emph{J. Graph
    Theory}, 1(1):19--25, 1977.

\bibitem{Ulam} S.~M. Ulam. \newblock \newblock {A collection of
    mathematical problems}. \newblock Interscience Tracts in Pure and
  Applied Mathematics, no. 8.  Interscience Publishers, New
  York-London, 1960.
  
\bibitem{WS} D.~B. West and H. Spinoza.
 \newblock Reconstruction from $k$-decks for graphs with maximum degree~2.
 \newblock \arxiv{1609.00284vi}, 2016.

\end{thebibliography}

\end{document}